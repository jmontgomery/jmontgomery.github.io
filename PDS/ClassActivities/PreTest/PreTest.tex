\documentclass[11pt]{article}
\usepackage{amsmath, geometry,hyperref, setspace}
\usepackage{url}
\thispagestyle{empty}
\begin{document}
\begin{center}
{\large    Statistical Computing\\
   Activity 1 }
\end{center}

\subsubsection*{Put your name at the top of this sheet. Write the R code to answer the following questions. Write the code and then show what the computer returns when that code is run.}

\flushleft
\begin{enumerate}
\item $e^2$ \vspace{12mm}
\item $((4)^5)^\frac{1}{8}$ \vspace{20mm}
\item $sin(\frac{\pi}{3}) \times (1 + tan(\frac{\pi}{3}))$ \vspace{12mm}
\item $\sqrt{14^3 - 6^\frac{3}{2}} $ \vspace{12mm}
\item $\mid-\ln(2\pi \times (\sqrt{e^9}))\mid$ \vspace{12mm}
\item $\sum_{i=5}^{50}i^{i-1}$ \vspace{12mm}
\item $\forall_{i}\in[1,50] \;\;\; \sqrt(i)$
\end{enumerate}

\newpage


\flushleft
\begin{enumerate}
\item You run a regression where your main explanatory variable is a binary indicator for Party ID (0=Democrat, 1=Republican) and you dv is a continuous measure of feelings towards President Trump (0-100 scale). The regression coefficient you get back your party variable is 6.25.  What does that mean?\vspace{10mm}
\item If someone asked you to cross-validate your regression model, what do they want you to do? \vspace{15mm}
\item You have an experiment where you asked each respondent to look at 6 news headlines and determine whether or not they are fake (a zero 1 outcome).  So for each person you have six rows.  Using tidyverse commands, how would you find the mean number of headlines each respondent thought were false .  Write the code as best you can. \vspace{15mm}
\item Write a SQL query to choose two variables names ``party'' and ``vote'' from an existing database called ``congress''. \vspace{10mm}

\item List three ways to improve the speed of an R function.  How would you go about evaluating the speed of a function? \vspace{15mm}

\item What is the Markdown code to create a bullet list?


\end{enumerate}



\end{document}
\newpage

\subsubsection*{Write the R code to answer the following questions.}


\begin{enumerate}
\item Calculate the sum $\sum_{j=1}^n j^2$ and compare with $n(n+1)(2n+1)/6$, for $n=200,~400,~600,~800$.
\item Using \texttt{rep()} and \texttt{seq()} as needed, create the vectors:
\begin{center}0 0 0 0 0 1 1 1 1 1  2 2 2 2 2 3 3 3 3 3 4 4 4 4 4\end{center}
\begin{center}1 2 3 4 5 1 2 3 4 5 1 2 3 4 5 1 2 3 4 5 1 2 3 4 5\end{center}
\begin{center}1 2 3 4 5 2 3 4 5 6 3 4 5 6 7 4 5 6 7 8 5 6 7 8 9\end{center}
\item Write the R code to calculate the standard sample variance formula
$$s^2 = \frac{1}{n-1} \sum_{i=1}^n(x_i-\bar{x})^2 $$
\item Look at the help file for the \texttt{sample()} function in R.
  Using only the help documentation, describe the output of the
  \texttt{sample()} function.
\item Change the sign of every odd number in x
\begin{verbatim}
x <- sample(-100:100, size = 100) 
\end{verbatim}
\item Take the dot product of x and y
\begin{verbatim}
x <- 1:100
y <- 100:1
\end{verbatim}
\item Use the \texttt{seq()} and \texttt{paste()} to create the vector called \texttt{varnames} containing

\begin{verbatim}
"Var1" "Var2" "Var3" "Var4" "Var5" "Var6"
\end{verbatim}
\item Remove the substring \texttt{``Var''} from the \texttt{varnames} vector
\item Recast the \texttt{varnames} vector into a numeric.
\item Subset the resulting vector \texttt{varnames} to be only odd numbers and make a new vector called  \texttt{varnames2} 
\item If I run the command
\begin{verbatim}
varnames - varnames2
\end{verbatim}

what calculation is being performed?

\end{enumerate}







\end{document} 