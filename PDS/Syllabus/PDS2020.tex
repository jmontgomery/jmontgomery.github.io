\documentclass[11pt]{article}
\usepackage{amsmath}
\usepackage{fullpage}
\usepackage{geometry}
\usepackage{setspace}
\usepackage{multirow}
\geometry{letterpaper, left=1in, right=1in, top=1in}
\usepackage{mathpazo}
\usepackage[pdftex]{hyperref}
\usepackage{booktabs}
\usepackage{multicol}

\begin{document}
\pagestyle{plain} 
\pagenumbering{arabic}

\begin{center}

\vspace{5pt}
\Large{Political Data Science (Beta) }\\
\Large{L32 4625}
\vspace{4 mm}

\small


\vspace{4 mm}

     \begin{tabular}{l }
       Tuesdays and Thursdays\\
      10:00-11:20AM \\
      Seigle 104
       \end{tabular}
\end{center}
%\vspace{4 mm}

\subsection*{Instructor Information}

\noindent Jacob M. Montgomery, Ph.D.\\
\noindent Assistant Professor, Department of Political Science\\
Office: Seigle 285\\
E-mail: jacob.montgomery@wustl.edu\\
Telephone: None \\
Office Hours: Wed. 10-12 and by appointment 

\section*{Course description}


Rapid improvements in computing power, advances in powerful
algorithms, and a massive increase in the availability of data has
created a new world of possibilities for businesses, governments,
policymakers, and decision-makers at every level. The goal of this
class is to make you a participant rather than a spectator in this new
data-driven world.  In particular, the goal is to give you real-world
skills and experience working with real data to answer real questions
about our political world.

This class is designed to achieve four broad objectives. First, it
aims to guide students as they learn the specifics of the R
programming language, a powerful statistical computing environment
widely used in the fields of political science, network analysis,
machine learning, and statistics.  Second, the class will also
introduce students to some the basic approaches to data analysis,
storage, and visualization in the machine learning/statistics
literatures.  So you are going to learn how to code. But more
importantly, you will learn how to acquire, manipulate, store,
visualize, and analyze data to answer real-world questions.

More broadly, this course aims to provide students some of
the foundational concepts and skills needed to engage in modern data
science.  No course can teach you \textit{everything} there
is to know about R, machine learning, and statistical science even as
it exists today.  Even more certainly no class can teach you everything you will need to use in your future career.  Some of the
tools that will be in wide use in ten years do not even exist today.
Thus, this course aims to give you the more foundational meta-skills
you need to teach \textit{yourself} the toolkit of the future.  Learning at this level will also better
equip you to understand results and analyses produced by others.

Finally, the end of the course will shift from the academic to the
practica.  Students will be tasked
with tackling a real world political problem using the skills they
have learned in the course. A major component of the course includes
learning how to plan and execute a collaborative, complex data science
project as well as how to effectively document and communicate the results.
 
\newpage

\subsection*{Learning objectives}

By the end of this course, you should be able to:

\begin{itemize}
\setlength\itemsep{0em}
\item Explain the basic components of the R working environment
\item Understand object-oriented programming (or at least R's version
  of this)
\item Understand the basic control functions, flow functions, and data
  structures of R
\item Functionalize complex and/or repetitive code
\item Clearly document code and develop a codebase on a collaborative platform
\item Read-in and write-out data and text of any format, including
  information collected online
\item Store complex data in a database
\item Create custom data visualizations
\item Fit basic machine learning models and be able to interpret the results
\item Create a basic website and webapp
\item Work collaboratively to plan and execute a question-driven data science project
\end{itemize}


\section*{Requirements and Evaluation}

Grading in this class will be based on the components described
below. \textbf{Late work will not be accepted without prior
  permission}. Makeup exams will not be given, and students who miss
exams will receive a score of 0 absent extraordinary circumstances.

\subsubsection*{Grading scale}
\vspace{2mm}

\begin{center}
\begin{tabular}{| l l | l l | l l | l l |}
\hline
Score & Grade & Score & Grade & Score & Grade& Score & Grade \\
\hline
$\ge$94 & A  & $\ge$83 & B & $\ge$ 73 &  C &$\ge$63 & D   \\
$\ge$90 & A-  & $\ge$80 & B- & $\ge$ 70 &  C- &$\ge$60 & D- \\
$\ge$87 & B+ & $\ge$77 & C+ & $\ge$ 67 &  D+& $<$60 & Fail \\
\hline

\end{tabular}
\end{center}



\subsection*{Peer assessments - 10\%}

You will be assigned into a team of 3-5
individuals.  You will work with this team throughout the semester on
in-class assignments and your final research project.  To help ensure
that all members of the team are actively contributing, students will
be asked to evaluate their teammates' contributions, effort, and
performance.  You will receive ungraded midterm evaluations from your
team to help you know how well you are doing and identify areas in
need of improvement. 


\subsection*{Problem sets and in-class work - 30\%}

\noindent \textit{Problem sets}, or homeworks, will be distributed
throughout the course (20\%). \textit{Unless otherwise specified},
these are individual assignments that you should prepare yourself,
though you may ask your colleagues for help. To be clear,
\textit{every single keystroke for these assignments should be your
  own}.  DO NOT COPY AND PASTE CODE FROM YOUR PEERS, THE INTERNET OR
ANYWHERE ELSE.  Please turn them in at the on the specified date
\textbf{at the beginning of class}. If you have a printing problem,
you are responsible for emailing it to me or the graduate TA before
class starts.  Each student's lowest homework grade will be dropped in
the final grade calculations.  This option should be reserved for
illness, family emergencies, broken alarm clocks, or other unforeseen
events.  No additional waivers will be granted.

 \vspace{.4cm}
 \noindent \textit{In-class assignments} will be completed during class
 with your team (10\%).  All members will turn in a single assignment at
 the end of class and will share their grade. However, \textbf{all
   absent students will receive a zero}.  Students missing more than
 five minutes of class time will be counted as absent.  Each student's
 two lowest in-class assignment grades will be dropped in the final
 grade calculations.  This option should be reserved for illness,
 family emergencies, broken alarm clocks, or other unforeseen events.
 No additional waivers will be granted.

\subsection*{Midterm exam - 25\%}
The midterm exam will be a take-home exam where you will be expected
to independently create an organized codebase to accomplish a specific
task. The exam will be due at the beginning of class on March 8.  Specific rules will be
explained at the time of the exam.


\subsection*{Project - 35\%}

\begin{itemize}
\item \textbf{Graduate Students}: After the midterm exam, teams will be
  assigned a specific programming task of interest either to myself or
  another faculty member in the department.  Working with your
  assigned teams, and under the close supervision of the faculty
  member, students will be responsible for planning, creating, and
  documenting an analysis  that meets the specified needs of the
  faculty member.  
\item \textbf{Undergraduates}: After the midterm exam, teams will
  consult with me to choose a project of their own for which they will
  generate a data science solution.  Working with your assigned teams,
  students will be responsible for planning, creating, and documenting
  a codebase to achieve the goal as chosen by the team.  This will
  include outlining the research goals, developing a plan, monitoring
  the progress of the team to specific benchmarks, and evaluating the
  final product.  
\end{itemize}

\noindent The course will culminate with turning over the results in
the form of a website or app to me for grading at the time of the
regularly scheduled final for this course.

\subsection*{Grades are final}

\textbf{No adjustments will be made to final grades under any circumstances
and no incompletes will be granted absent extraordinary
circumstances}.  

\section*{Class policies}

\subsection*{Teaching Assistant}

There is one graduate teaching assistant. She will work closely in
conjunction with Professor Montgomery on all issues of grading, but
all grading decisions will be mine.  
\vspace{.2cm}

\begin{center}
\begin{small}
\begin{tabular}{c}
Dominique Lockett \\
DLOCKETT@WUSTL.EDU\\
Office Hours: Tuesdays, 1pm-3pm\\
Office location: Seigle 278
% Mr. David Carlson&  Mr. Jonathan Homola\\
%  carlson.david@wustl.edu & homola@wustl.edu\\
% Office Hours: TBA & Office Hours: TBA\\
% Office: Seigle 276 & Office: Seigle 277\\
\end{tabular}
\end{small}
\end{center}

There is also one undergraduate teaching assistant. She will work on the online course book and hold office hours.
\vspace{.2cm}

\begin{center}
\begin{small}
\begin{tabular}{c}
Mariah Yelenick \\
MYELENICK@WUSTL.EDU\\
Office Hours: Tuesdays, 4pm-6pm\\
Location: Whispers\\
(if you can't find her, message her on Slack)
\end{tabular}
\end{small}
\end{center}

\subsection*{Communications/Slack/Github}

Email. is. just. the. worst.  Using my data science skills I estimate that I get 1,385.3 emails/day (this is made up). For this course I have made a Slack workspace at \url{pds-class.slack.com}.  Please install the App on your computer and/or phone and add this workspace.  All class communications will occur here. We will also make individual channels for each problem set and groups are encouraged to make their own channels to facilitate communications.  

You are, of course, allowed to communicate with me or the TA privately through Slack or email.  However, unless it is confidential, I prefer that questions appear on the general channels so all students can see the question, the answer, and maybe offer solutions themselves.

Later in the semester we will also introduce you to Git and GitHub.  GitHub is a powerful platform for collaborative and open data science and once you get the hang of it all problem sets, exams, and projects will occur here.  It also offers nice solutions for hosting basic webpages.  So go ahead and sign yourself up for a GitHub account now if you don't already have one.



\subsection*{Technology in the classroom}

You will frequently make use of computers in this course. Plan to
bring your laptop to class and have it ready to use each day.  Please
be respectful to your instructors and your peers by using your
computers only for class-related purposes.  Please put your phone away
before class starts and don't bring it out.  If I find you tweeting,
playing games, watching sports, shopping, moderating Reddit, whatever
TikTok is, or curating your Instagram feed, I will immediately stop
class to have a discussion with you about what aspect of the current
class session you find boring or uninteresting.  You have been warned. 

\subsection*{Academic Honesty} 

Cheating and plagiarism will not be tolerated.  I strongly encourage
you to review the University's policies regarding academic honesty,
which you can read at:
\url{http://www.wustl.edu/policies/undergraduate-academic-integrity.html}.

In general, if you have any question, please feel free to ask
your TA or Professor Montgomery. Specific rules for this course:
\begin{itemize}
\item You may work together on homework in small groups, but you
  should each prepare your answers separately unless otherwise instructed.
\item The homeworks and in-class work are ``open book'' and ``open
  notes.''  
\item You are to consult \textit{only} with Professor Montgomery or
  a TA during exams.
\item See the discussion of the rules for the problem sets above.
\end{itemize}

\noindent All cases of cheating or plagiarism will be referred to Washington
University's Committee on Academic Integrity. If the Committee on
Academic Integrity finds a student guilty of cheating, then the
penalty will be (without exception) automatic failure of the course.

\subsection*{Students with disabilities}

Students with disabilities enrolled in this course who may need
disability-related classroom accommodations are encouraged to make an
appointment to see me before the end of the second week of the
semester.  All conversations will remain confidential. Please also
arrange to have the required documentation sent to me for any
accommodations \emph{at least two weeks prior to the first exam.}

\subsection*{Religious observances}

Some students may wish to take part in religious observances that
occur during this semester. If you have a religious observance
that conflicts with your participation in the course, please meet with
me \emph{before the end of the second week of the semester} to discuss
accommodations.


\subsection*{Reporting and accommodations for sexual assault}

If a student discusses or discloses an instance of sexual assault, sex discrimination, sexual harassment, dating violence, domestic violence or stalking, or if a faculty member otherwise observes or becomes aware of such an allegation, they will keep the information as private as possible, but as a faculty member of Washington University, they are required to immediately report it to the Department Chair or Dean or directly to Ms. Jessica Kennedy, the University’s Title IX Director, at (314) 935-3118, jwkennedy@wustl.edu.  Additionally, you can report incidents or complaints to the Office of Student Conduct and Community Standards or by contacting WUPD at (314) 935-5555 or your local law enforcement agency. 

The University is committed to offering reasonable academic accommodations (e.g., no contact order, course changes) to students who are victims of relationship or sexual violence, regardless of whether they seek criminal or disciplinary action.  If you need to request such accommodations, please contact the Relationship and Sexual Violence Prevention Center (RSVP) at rsvpcenter@wustl.edu or 314-935-3445 to schedule an appointment with an RSVP confidential, licensed counselor. Information shared with counselors is confidential. However, requests for accommodations will be coordinated with the appropriate University administrators and faculty.  




\section*{Course materials}

\subsection*{Textbooks}


In addition to assigned readings that will be posted on Blackboard,
the following books are required.  You may be able to find free
versions online or through the library. \singlespacing

\begin{quote}
\noindent de Vries, Andrie and Joris Meys. 2015.  \textit{R for
  Dummies} (2nd Edition). Wiley. 
\vspace{.2cm}

\noindent Wickam, Hadley and Garrett Grolemund. \textit{R for Data Science}. OReilly.  
\url{https://r4ds.had.co.nz/}

\vspace{.2cm}

\noindent Wickham, Hadley.  \textit{Advanced R}. CRC Press. \url{https://adv-r.hadley.nz/}
\vspace{.2cm}

\noindent Wickham, Hadley.  \textit{Mastering Shiny}. O'Reilly. \url{https://mastering-shiny.org/}
\vspace{.2cm}

\noindent Baumer, Benjamin, Daniel K. Kaplan, and Nicholas J. Horton. \textit{Modern Data Science with R}. CRC. 

\noindent Nolan, Deborah and Duncan Temple Lang.  \textit{XML and Web Technologies for Data Sciences with R}.  Springer.


Additional readings/links/resources provided at \url{politicaldatascience.com}
\vspace{.2cm}





\end{quote}





% \noindent The following books are suggested for additional readings.  

% \begin{quote}
% \vspace{.2cm}

% \noindent Braun, W. John and Duncan J. Murdoch.  2007.  \textit{A First Course in Statistical Programming with R}.  Cambridge.

% \vspace{.2cm}

% \noindent Jones, Owen,  Robert Maillardet, and Andrew Robinson.  2009.
% \textit{Introduction to Scientific Programming and Simulation Using R}.  Chapman and Hall (CRC)
% \end{quote}



\subsection*{Software and hardware}

You will be using the R statistical package
(\url{http://www.r-project.org/}). While R is available for every
computing platform, some of the more advanced tasks performed in this
class will be taught based on the assumption that you are working on a
Mac or Windows machine.  Linux users will need to work with me to find
solutions on your machine.  Note: \textit{Once you are set up with R,
  R-Studio, and Git, do NOT update your operating system (OS) for the
  remainder of the semester.  If you are considering updating your OS,
  do it now}.



\section*{Plan of the course}

The basic outline of the course is divided into four components. In
Section 1, we will introduce the R computing environment and develop
some basic and advanced skills and topics.  In this portion of the
class, we will begin with short lectures and discussions of the
assigned readings. To encourage engagement with these materials,
students will be asked to accomplish assigned programming tasks both
inside and outside of the class period. 

In Section 2, we will move onto making use of existing software
packages that allow you to acquire, manipulate, and visualize
data. Largely, this will come in the form of learning to use some of
the functionality offered in the \textit{tidyverse} of R packages.

In Section 3, we will introduce several basic approaches to analyzing
data.  The goal will be to provide an intuition about these models as
well practical advice as to how to fit, interpret, and evaluate performance.  

In Section 4, we will move from the abstract to the applied as the
class takes on several real-world statistical programming challenges.
Under the close supervision of me (, each team will work on a project
requiring the development of a complex set of code to answer a
real-world question about our political world.  As part of this
Section, several class periods will be dedicated to covering some
advanced topics that will be needed to execute their projects.  Likely
topics include webscraping, API access, SQL databases, website
creation/hosting, and Shiny.  By the end of this Section, each team
will be able to produce a complete data science project that will pull
together skills in coding, data acquisition, data visualization,
machine learning, and website/app development.

\subsection*{Very tentative Schedule}

All told, this is an ambitious project that will require a substantial
intellectual engagement from each student. It will also require
\textit{flexibility} since the course may evolve -- perhaps
substantially -- during the semester in response to the needs of the
project, our results, and issues raised by students.  Y'all are coming from a lot of different backgrounds so I will need to adjust pacing and structure in response.

In particular, you will be expected to work with your teams throughout
the semester both inside and outside the class. You will be involved
in many collaborative projects in your career, so consider building a
positive working dynamic within your team to be part of the
assignment.  I also expect that students be quick to inform me and/or
the TA when assignments seem vague, overly difficult, confusing, or
incomplete.  \textbf{The schedule below should be viewed as no more than
suggestive}.




\begin{small}
\begin{center}
\begin{tabular}{p{1.5cm} p{4cm} p{3.5cm} p{3cm} p{4cm}}
  \toprule
  Date & Topic & Reading & Assignments & Notes  \\
  \midrule
\textbf{Section 1} \\
1/14 & \texttt{``Hello world''}  & DVM Chpt 1-2, Appendix& & Pre-test\\
& & \textit{Advanced R} Chapter 1\\
\\
1/16 & Objects & DVM Chpt 4,5  & & PS 1 dist.\\
& & \textit{Advanced R} Chapter 2\\
\\
1/21 & Classes, Characters, Logicals & DVM Chpt 7 \\
& & \textit{Advanced R} Chpts. 3, 4   \\
\\
1/23 & Data structures & \\
& & \\
1/28 & Control/flow  & DVM Chpt 9 & PS 1 due \\
\\
1/30 & Functions  & DVM Chpt 8 &  & Teams assigned\\
& & \textit{Advanced R}  Chpt 6 \\
\\ 
2/4 & Version control & DVM Chpt 11\\
& Documentation & \textit{Advanced R}  Chapter 5, 27 \\
  \\
2/6 & Debugging & \textit{Advanced R}  Chpt 22-24 \\
& Testing/evaluating  & \\
\textbf{Section 2} \\
2/11 & \texttt{ggplot}  & \textit{R4DS} Chpt 3 &\\
  \\
2/13 & \texttt{dplyr}, \texttt{tidy} &  \textit{R4DS} Chpt 5, 11-12 & PS 2 due \\
  \\
2/18 & Relational data & \textit{R4DS} Chpt 13 \\
  \\
2/20 & Text-as-data  & \textit{R4DS} Chpt 14 & & Practice midterm dist.\\
& & \textit{MDSwR} Chpt 15\\
  \\
2/25 & pipes, \texttt{map}, \texttt{walk}  &  \textit{R4DS} Chpt 18, 21 \\
  \\
2/27 & Functional programming &  \textit{Advanced R} Chpt 9-11 & PS 3 due & Not covered in exam \\
& \texttt{purrr} \\
\end{tabular}
\end{center}
\end{small}


\begin{small}
\begin{center}
\begin{tabular}{p{1.5cm} p{4cm} p{3.5cm} p{3cm} p{4cm}}
  \toprule
  Date & Topic & Reading & Assignments & Notes  \\
  \midrule
\textbf{Section 3} \\
3/3 & Machine learning intro  & & & Midterm distributed\\
\\
3/5 & Supervised learning 1& \textit{MDSwR} Chpt 7 &  Midterms due& Midterm evals\\
\\
3/17 & Supervised learning 2& \textit{MDSwR} Chpt 8   & & PS 4 dist.\\
\\
3/19  & Unsupervised learning 1& \textit{MDSwR} Chpt 9.1 & & Also topic models \\
\\
3/24 & Unsupervised learning 2 &\textit{MDSwR} Chpt 9.2 & & \\
\\
3/26  & Causality 1: A/B testing & Online content & PS 4 Due & PS 5 Dist. \\
\\
3/31 & Causality 2 : Observational strategies &  Online content \\
\\
\textbf{Section 4} \\
4/2 &  Interactive visualizations & \textit{Mastering Shiny} 1-5 \\
& & \textit{MDSwR} Chpt 11 \\
\\
4/7 & Webscraping 1 & \textit{XML} Chpt 1-3\\
\\
4/9 & Webscraping 2 & \textit{XML} Chpt 4-5 & PS 5 due \\
\\
4/14 & SQL and friends  & \textit{MDSwR} Chpt 12, 13\\
\\
4/16 & Github pages & & & Dom (MPSA)\\
\\
4/21 & Using APIs & \textit{XML} 8,10 \\
\\
4/23 & TBD  &  & & Post-test\\
\\
5/4 & FINAL PROJECTS DUE
\end{tabular}
\end{center}
\end{small}


\newpage
\subsection*{Peer evaluation form (end of semester)}
\pagestyle{empty}

Name/team \#: \\

\noindent Please assign scores that reflect how you really feel about the extent to which the other members of your team contributed to your learning and/or your teamÕs performance. This will be your only opportunity to reward the members of your team who worked hard on your behalf. (Note: If you give everyone pretty much the same score, you will be hurting those who did the most and helping those who did the least.)\\

\noindent \textbf{Instructions:} In the space below, please rate each of the other members of your team. Each member's peer evaluation score will be the average of the points they receive from the other members of the team. To complete the evaluation you should: 1) List the name of each member of your team in the alphabetical order of their last names and, 2) assign an average of ten points to the other members of your team and, 3) differentiate some in your ratings; for example, you must give at least one score of 11 or higher (maximum = 15) and one score of 9 or lower.\\

\begin{center}
\begin{large}
\begin{onehalfspacing}
\begin{tabular}{| p{.5cm} p{5.5cm} p{2.5cm} |}
\hline
&\textbf{Team member} & \textbf{Score} \\

1. & &  \\

2. &  &  \\

3. &  &  \\

4. &  &\\
\hline
\end{tabular}
\end{onehalfspacing}
\end{large}
\end{center}

\subsubsection*{Additional feedback}

Please briefly describe the reasons for your highest and lowest ratings in the space below. These comments will be shared anonymously. Note: Your comments should be descriptive, not evaluative; as clear and specific as possible; phrased in constructive terms; and focused on areas in which the student has made especially valuable contributions or could improve in the future. \\

\noindent Reason(s) for your highest rating(s): \\

\vspace{2cm}

\noindent Reason(s) for your lowest rating(s):




\end{document}
