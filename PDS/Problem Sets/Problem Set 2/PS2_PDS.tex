% Options for packages loaded elsewhere
\PassOptionsToPackage{unicode}{hyperref}
\PassOptionsToPackage{hyphens}{url}
%
\documentclass[
]{article}
\usepackage{lmodern}
\usepackage{amssymb,amsmath}
\usepackage{ifxetex,ifluatex}
\ifnum 0\ifxetex 1\fi\ifluatex 1\fi=0 % if pdftex
  \usepackage[T1]{fontenc}
  \usepackage[utf8]{inputenc}
  \usepackage{textcomp} % provide euro and other symbols
\else % if luatex or xetex
  \usepackage{unicode-math}
  \defaultfontfeatures{Scale=MatchLowercase}
  \defaultfontfeatures[\rmfamily]{Ligatures=TeX,Scale=1}
\fi
% Use upquote if available, for straight quotes in verbatim environments
\IfFileExists{upquote.sty}{\usepackage{upquote}}{}
\IfFileExists{microtype.sty}{% use microtype if available
  \usepackage[]{microtype}
  \UseMicrotypeSet[protrusion]{basicmath} % disable protrusion for tt fonts
}{}
\makeatletter
\@ifundefined{KOMAClassName}{% if non-KOMA class
  \IfFileExists{parskip.sty}{%
    \usepackage{parskip}
  }{% else
    \setlength{\parindent}{0pt}
    \setlength{\parskip}{6pt plus 2pt minus 1pt}}
}{% if KOMA class
  \KOMAoptions{parskip=half}}
\makeatother
\usepackage{xcolor}
\IfFileExists{xurl.sty}{\usepackage{xurl}}{} % add URL line breaks if available
\IfFileExists{bookmark.sty}{\usepackage{bookmark}}{\usepackage{hyperref}}
\hypersetup{
  hidelinks,
  pdfcreator={LaTeX via pandoc}}
\urlstyle{same} % disable monospaced font for URLs
\usepackage[margin=1in]{geometry}
\usepackage{graphicx,grffile}
\makeatletter
\def\maxwidth{\ifdim\Gin@nat@width>\linewidth\linewidth\else\Gin@nat@width\fi}
\def\maxheight{\ifdim\Gin@nat@height>\textheight\textheight\else\Gin@nat@height\fi}
\makeatother
% Scale images if necessary, so that they will not overflow the page
% margins by default, and it is still possible to overwrite the defaults
% using explicit options in \includegraphics[width, height, ...]{}
\setkeys{Gin}{width=\maxwidth,height=\maxheight,keepaspectratio}
% Set default figure placement to htbp
\makeatletter
\def\fps@figure{htbp}
\makeatother
\setlength{\emergencystretch}{3em} % prevent overfull lines
\providecommand{\tightlist}{%
  \setlength{\itemsep}{0pt}\setlength{\parskip}{0pt}}
\setcounter{secnumdepth}{-\maxdimen} % remove section numbering

\author{}
\date{\vspace{-2.5em}}

\begin{document}

\begin{center}
{\Large{\textbf{Problem Set 2}}} \\
\vspace{4 bp}
Due February 13, 10:00 AM (Before Class) \\
\end{center}

\section*{Instructions}
\begin{enumerate}
\item The following questions should each be answered within an R script. Be sure to provide many comments in the script to facilitate grading. Undocumented code will not be graded. Once your script is finished, please email Dominique at dlockett@wustl.edu.
\item You may work in teams, but each student should develop their own R script. To be clear, there should be no copy and paste. Each keystroke in the assignment should be your own.
\item If you have any questions regarding the Problem Set, contact the TA or use her office hours.
\item For students new to programming, this may take a while. Get started.
\end{enumerate}

\#make a function that calls their function and performs a test

\textbackslash begin\{enumerate\}

\section*{Working with data in R}

\vspace{1cm}
\item

Calculate the following probabilities:

\item

Probability that in 60 tosses of a fair coin the head comes up:

\begin{enumerate}
\item 15, 20, or 30 times
\item less than 20 times
\item between 20 and 30 times
\end{enumerate}

\begin{itemize}

\item Write a for loop that does 1000 simulations of where two fair dice are rolled.
\begin{itemize}
\item Write the loop such that if the two dice total to values 8,9,10,11,12 the game ends immediately
\item If the first roll does not equal one of those five values continue to roll the dice until you roll either a 2 or a 6
\item what is the average number of dice casts per game
\end{itemize}\end{itemize}

\item

game1 \textless- list(``Game 1'' =cbind(3 ,2 ),``Game 2'' = cbind(1,2),
``Game 3'' =cbind(8,4), ``Game 4''= cbind(2, 1), ``Game 5'' = cbind(4,
6))

colname \textless- c(``Player 1'', ``Player 2'') for (i in
seq\_along(game1))\{ colnames(game1{[}{[}i{]}{]}) \textless- colname \}

Run the above code. The game object includes the results of five
different games among 2 players. Write a for loop which returns ``Win!''
if Player 1 wins the game and write a function which returns ``Lose :(''
if Player two wins.

game2 \textless- list(``Game 1'' =cbind(3 ,3 ),``Game 2'' = cbind(NA,2),
``Game 3'' =cbind(8,4), ``Game 4''= cbind(2, NA), ``Game 5'' = cbind(4,
4), ``Game 5'' = cbind(3, 4)) colname \textless- c(``Player 1'',
``Player 2'') for (i in seq\_along(game2))\{
colnames(game2{[}{[}i{]}{]}) \textless- colname \}

Now, run this new code and add to your for loop a function that returns
``Draw!'' if player 1 and player 2 have a tie and also include an
argument that returns the statement ``Warning, there were not enough
values in this game'' if there is an NA in the eithe players' values.
\textbackslash end\{enumerate\}

\end{document}
